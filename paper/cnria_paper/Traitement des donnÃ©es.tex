\section{Traitement des données}\label{Traitement des données}
Dans cette section, nous détaillons le processus de préparation des données suivi pour l’obtention d’un jeu de données sur le paludisme pour la phase de prédiction. Nous commençons par présenter les techniques de nettoyages et de normalisation des données utilisées.
\subsection{Nettoyage et labélisation des données}
Dans le but de mettre en place un modèle de prédiction efficace des cas de paludisme au Sénégal, nous nous sommes appuyés sur un jeu de données de patients réel pour la validation. Le jeu de données a été extrait du Grand Magal de Touba de 2016 [19 ]. Environ 4-5 millions d'individus se réunissent chaque année dans la ville sainte de Touba, au Sénégal, lors de la cérémonie religieuse du Grand Magal. Plusieurs points de santé sont établis lors de cet événement religieux; chaque point reçoit chaque jour des centaines de patients, dont certains atteints de paludisme. Les données de patients que nous utilisons ici ont été recueillies manuellement à partir des registres de ces points puisqu’ aucun système de gestion électronique de la santé n’existe. En détail, l’ensemble des données comprend des milliers d’enregistrements de patients comportant chacun 16 caractéristiques. Certaines de ces caractéristiques (également appelées fonctionnalités) comprennent des données personnelles sur le patient, mais aussi les signes et les symptômes du patient signalés par le médecin qui a pris le patient en charge. Les autres attributs décrivent des données cliniques telles que des informations sur le diagnostic final du médecin (la maladie dont souffre le patient), le résultat du test de diagnostic rapide (TDR) et le statut (c.-à-d. admission, décédé ou mis sous observation) du patient. Pour des raisons de confidentialité et certaines restrictions d’utilisation des données, nous avons ignoré les données personnelles sur le patient pendant ce travail. En raison du fait que les données des patients ont été collectées manuellement dans des registres, nous avons constaté de nombreuses incohérences telles que fautes d'orthographe, mêmes valeurs d'attributs avec des écritures différentes (par exemple, «DIARRHEE INFECTIEUSE ”et“ INFECTIEUSE DIARRHE ”), et des attributs à valeurs multiples (par exemple la colonne signes et symptôme). Nous utilisons le logiciel OpenRefine [13 , 1] pour d’abord nettoyer  puis normaliser les valeurs dans l'ensemble des données  des patients. OpenRefine (Google raffiné) est un puissant outil open source qui permet aux chercheurs ou aux scientifiques d’accomplir l’activité de criblage des données, c’est-à-dire  de travailler avec des données en désordre: les nettoyer; les transformer d'un format à un autre; et les compléter avec des services Web et des données externes. Nous avons utilisé les méthodes suivantes fournies par OpenRefine pour prétraiter notre jeu de données brutes.
\begin{itemize}
\item \textbf{Text filter function:} le filtre de texte permet d’explorer les valeurs des attributs, de les nettoyer et d’identifier ceux qui peuvent avoir de nombreuses variantes.
\item \textbf{Transform functions:} OpenRefine fournit deux fonctions de transformation différentes: les fonctions de transformations prédéfinies (preset transformations functions) pour la résolution de problèmes de mise en forme triviaux tels que le rognage des espace blancs et des fonctions de transformations avancées (Advanced transformations) basées sur le langage (GREL) de OpenRefine pour normaliser les données par lots ou les fractionner. Cette deuxième classe de formations est très utile, en particulier lorsque le nombre de valeurs de données à normaliser est très important (faire la même tâche manuellement prendrait beaucoup de temps et serait sujette aux erreurs). Par exemple, GREL permet d’utiliser une expression régulière simple pour toutes les variantes d'un symptôme dans la colonne Symptôme.
\item \textbf{Cluster and edit function:} l'option de groupage dans OpenRefine fournit également aux utilisateurs des méthodes pour fusionner et normaliser les variations dans l'ensemble de données. La puissance de cette fonction est qu'il est capable de détecter automatiquement les petites variations de données qui suivent un certain modèle.
\end{itemize}
En ce qui concerne le cas particulier des attributs à valeurs multiples tels que les colonnes symptôme et diagnostic de notre jeu de données brutes, nous les avons divisés en plusieurs valeurs dans des colonnes distinctes. En effet, dans le jeu de données brutes, des informations telles que les symptômes dont un patient donné souffre ont été stockés dans une seule colonne, séparée par le caractère spécial '+', par exemple \textquote{DOULEUR ARTICULAIRE}+\textquote{DOULEUR PELVIENNE} +\textquote{VOMISSEMENTS}.
Après cette étape de nettoyage et de normalisation des données, nous avons procédé à l'extraction des caractéristiques du paludisme.
\subsection{Sélection  des caractéristiques du paludisme}
Pour bien étudier le paludisme, il faut disposer d’un ensemble de données d’un  patient comprenant les principales caractéristiques de la maladie.  Malheureusement, certaines de ces caractéristiques du paludisme n’étaient pas explicitement spécifiées dans notre jeu de données brutes. En conséquence, nous avons déduit douze nouveaux attributs qui décrivent mieux les signes et les symptômes du paludisme selon les experts du monde de la santé. Ces nouveaux attributs sont: \emph{manque d’appétit}, \emph{fatigue}, \emph{fièvre}, \emph{céphalalgie}, \emph{nausée}, \emph{arthralgie}, \emph{troubles digestifs}, \emph{vertiges}, \emph{frissons}, \emph{myalgie}, \emph{diarrhée} et \emph{douleurs abdominales}. Nous avons ensuite ajouté les nouveaux attributs à notre jeu de données et transformé ce dernière en remplissant la valeur de chaque nouvel attribut en fonction de la liste des signes et des symptômes signalés pour chaque patient.
Le Diagnostic est le résultat des signes et symptômes confirmé par un examen médical en général. Dans la colonne Diagnostic, on voit plusieurs conclusions différentes. Ainsi tout diagnostic autre que le paludisme est remplacé par la classe non paludisme
À cette étape de notre processus de préparation de données, nous avons créé un jeu de données patient contenant les caractéristiques requises pour le paludisme. Cependant, notre jeu de données n'était pas encore complet ni prêt à cause de l'absence de valeurs. Enfin, nous avons complété notre ensemble de données en utilisant une approche d’imputation des données robuste
\subsection{ Imputation de données manquantes}
Comme le montre le tableau 1, nous avons observé de nombreuses valeurs manquantes dans notre jeu de données, affectant ainsi la majorité des attributs de données. Ces valeurs manquantes ne doivent pas être ignorées car les données d’autant plus que l'exhaustivité et la qualité sont très importantes pour traiter un problème de prédiction; ceci pourrait avoir un impact négatif sur la précision de nos prédictions et devrait être traité de manière appropriée. Il faut noter que l'apprentissage automatique repose sur un ensemble de données complet. Les sources et les types de valeurs manquantes peuvent être variés [ 22] . Dans notre contexte, les manquements ne sont pas complètement aléatoires et peuvent être dus à une connaissance incomplète des données du patient, au fait que le personnel médical ne spécifie pas une valeur d’attribut lorsqu’elle n’est pas observée, ou à une difficulté pour les patients à décrire correctement certaines informations (liées par exemple aux signes ou aux symptômes de leurs maladies) au moment du diagnostic. Puisqu'ils pourraient avoir une certaine relation entre les valeurs d'attribut pour le même patient, voire une corrélation entre les dossiers des patients, nous décidons de résoudre notre problème de valeurs manquantes en utilisant des algorithmes d'imputation au lieu de choisir des valeurs arbitraires ou de supprimer des enregistrements avec valeurs manquantes.
L’imputation des données est souvent utilisée dans le domaine de l’apprentissage automatique pour traiter les erreurs d’informations. De nombreux algorithmes ont été proposés dans la littérature [18 , 22], dépendant de la nature de l’absence ou du type de données. MissForest [21] a été prouvé très efficace en présence de divers types de données simultanément comme dans notre cas (par exemple données numériques, chaîne, données catégorielles, etc.). L'algorithme missForest s'appuie sur 
RandomForest qui est une méthode de prédiction non paramétrique capable de traiter des données mixtes et qui permet des effets de régression interactifs et non linéaires. Un tel algorithme d’imputation vise à traiter tous jeu de données d’information en minimisant (dans la mesure du possible) les présomptions sur les aspects structurels des données. Étant donné un ensemble de données d’information, missForest résout le problème de données manquantes en utilisant un schéma d'imputation itérative aléatoire sur les valeurs observées dans une première étape, suivie de la prédiction des valeurs manquantes puis procédant de manière itérative jusqu'à la convergence.
Nous avons appliqué missForest à notre jeu de données de patient du paludisme obtenu du paludisme précédemment  en utilisant le logiciel Python [2]. 
Nous étudierons et prouverons dans la section 5 précisions de la prédiction faite par l’algorithme à l’aide la NRMSE (normalized root mean squared Error).


