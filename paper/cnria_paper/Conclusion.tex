\section{Conclusion}\label{conclusion}
Dans cet article, nous avons étudié le problème de la prédiction de la présence ou non de paludisme chez un patient considéré comme atteint dans le contexte du Sénégal et en utilisant des techniques d’apprentissage automatique.
Pour résoudre ce problème, nous avons d’abord présenté une méthode de préparation de données qui permet de nettoyer, normaliser et imputer les valeurs manquantes à partir d'un jeu de données réel en utilisant des outils et des algorithmes efficaces. Nous avons également introduit un moyen d’extraire les fonctionnalités qui caractérisent le paludisme. Nous avons ensuite proposé un modèle de prédiction basé sur la régression logistique pour déterminer l'apparition du paludisme. La performance d’un tel modèle a été démontrée à travers de nombreuses expérimentations sur le monde réel et un ensemble de données semi-synthétiques. Comme perspective de recherche, nous prévoyons d’abord d’inclure une prévalence facteur dans notre fonction de prédiction afin d’améliorer sa précision. Deuxièmement, nous allons utiliser d'autres modèles de classification binaire tels que Support Vector Machine (ou SVM en abrégé) et comparer leurs résultats à ceux obtenus avec le modèle basé sur la régression logistique.
