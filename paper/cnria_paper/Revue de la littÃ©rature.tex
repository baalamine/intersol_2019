Dans cette section, nous résumons l’état de la recherche sur le paludisme en général, et en particulier l’utilisation des techniques d’apprentissage automatique pour aborder les différents aspects liés à l’un des principaux problèmes de santé dans le monde, notamment le paludisme
Comme on le sait bien, le paludisme est causé par la piqure de l’anophèle femelle, dont la plus dangereuse espèce est le Plasmodium falciparum. De nombreux travaux préliminaires ont été ensuite consacrés à l’étude de l’évolution et de la répartition du moustique responsable, principalement dans le but de détecter ou de diagnostiquer la gravité de la maladie par rapport à un patient infecté donné [10 , 5]. Les recherches récentes sur le paludisme ont largement adopté le machine learning et ont démontré sa capacité à résoudre divers aspects de la maladie. La plupart de ces techniques basées sur le machine learning reposent sur l'analyse des données sanguines obtenues à partir de captures d'écran microscopiques de haute définition comme dans [12 ]. Les auteurs dans [12] proposent un algorithme d'apprentissage non supervisé qui détecte et détermine les types de cellules sanguines infectées. L’approche de prédiction utilisée consiste à quantifier la quantité de parasite plasmodium dans un frottis sanguin. Dans la même intuition de recherche d’exploitation du sang. ‘The Jordan-Elman neural networks classifier’ introduit dans [ 7] ,permet rapidement déterminer l'occurrence du paludisme et son niveau de gravité également. Elle est basée sur une analyse des caractéristiques des données sanguine des patients par le réseau de neurones. Toujours en utilisant le machine learning, DIAZ et al. ont proposé dans [ 9] un algorithme semi-supervisé permettant de quantifier et de classer les érythrocytes infectés par les parasites du paludisme à travers des images microscopiques. L'originalité de cette méthode vient de son efficacité même en présence de fines pellicules de sang infecté par le Plasmodium falciparum pour la quantification et de classification des parasites plasmodium infectés. Outre les données sanguines, des enregistrements de signes et de symptômes des patients ont également été utilisés pour étudier le paludisme avec les méthodes  de machine learning. En effet, une approche basée sur les arbres de décision a été proposée au Nigeria [ 23] pour prédire la survenue du paludisme à partir des données de diagnostic. Cependant, un arbre de décision souffre de diverse limite en tant que classificateur. En effet, il peut facilement sur-adapter ou peut être extrêmement sensible aux petites variations dans les données. Quand bien même nous nous appuyons et sur les signes et sur les symptômes, le modèle de prédiction dans [ 23] diffère du nôtre sur de nombreuses facettes: notre modèle est construit sur la régression logistique et est entrainé  en utilisant également les informations du test de diagnostic rapide. De plus, nous appliquons notre méthode dans le contexte de patients vivant au Sénégal. Un exemple du travail cité précédemment et qui a utilisé la régression logistique est celui de Farida et al. dans [ 3]. La régression logistique y est utilisée pour la sélection des attributs afin de construire des arbres de décision stables. Les arbres de décision sont ensuite utilisés pour prédire les critères de gravité du paludisme dans le contexte afghan.
Dans la lignée des travaux appliquant l’apprentissage automatique, dans [ 16] , Pranav et al. proposent un agent d'apprentissage par Reinforcement Learning  (RL) capable de prédire la probabilité qu'un individu présente un résultat positif au test du paludisme en posant des questions sur leur ménage. Cet agent est un Deep Q-network  RL qui apprend une politique directement à partir des réponses aux questions, avec une action définie pour chaque question de sondage possible et pour chaque classe de prédiction possible. En outre, une classification fondée sur des règles statistiques  améliorées et permettant de diagnostiquer le paludisme a été proposé dans [6].Un prototype correspondant intégrant les règles et les modèles statistiques a été mis en place; L’objectif principal de l’étude était de développer un prototype statistique permettant de réaliser un diagnostic clinique du paludisme, compte tenu de ses effets indésirables sur l’ensemble des soins de santé. Toutefois, son traitement reste très coûteux pour la majorité des patients.
À notre connaissance, il s’agit du premier travail au Sénégal qui tente de fournir un modèle de prédiction du paludisme à partir des données des patients.

