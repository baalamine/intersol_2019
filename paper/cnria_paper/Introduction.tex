
\section{Introduction}\label{Introduction}
Le paludisme est l’une des maladies les plus meurtrières au monde, en particulier dans les pays d’Afrique subsaharienne tels que le Sénégal. Le paludisme est causé par des microorganismes unicellulaires parasites appartenant au groupe Plasmodium; c'est une maladie infectieuse transmise à l'homme par les piqûres de moustiques Anophèles femelles infectées. Une personne atteinte de paludisme peut présenter des symptômes tels que : fièvre, fatigue, vomissements et maux de tête. Dans sa forme sévère, la maladie peut causer une peau jaune, des convulsions, le coma et  la mort.
\subsubsection{Problématique et Motivations.}
Selon le dernier rapport \cite{Wh17}  sur le paludisme dans le monde, publié en novembre 2017 par l’OMS, il y a eu 216 millions de cas de paludisme en 2016. On constate une hausse significative comparé aux 211 millions de cas enregistré en 2015.
 On estime à 445 000 le nombre de décès dus au paludisme en 2016, un chiffre similaire à celui de l’année précédente (446 000) malgré les efforts colossaux  consentis par les états et organismes non gouvernementaux dans le cadre de l’amélioration des services de santé et des stratégies de sensibilisation en  particulier dans les régions les plus affectées.
Une analyse profonde des statistiques ci-dessus montre le paludisme demeure toujours un facteur majeur de mortalité en Afrique. En effet la région OMS de l'Afrique supporte une part disproportionnée de la charge mondiale du paludisme. En 2016, 90\% des cas de paludisme et 91\% des décès dus à cette maladie sont survenus dans cette région. Plus précisément  80\% de la charge de morbidité due au paludisme pesaient sur une quinzaine de pays tous situés en Afrique subsaharienne, sauf l’Inde. Cela démontre que le paludisme reste un véritable fléau dans les pays d'Afrique subsaharienne et que le Sénégal n'est pas du tout épargné. Dans cette  étude nous proposons une approche  efficace pour prédire le paludisme en utilisant le machine learning (apprentissage machine) lors qu’un patient vient se faire consulte. A partir des signes et des symptômes du patient mais aussi du Test de Diagnostic Rapide (TDR) notre solution permettra de dire avec précision si un  patient souffre du paludisme ou non. 
Le paludisme est un grave problème au Sénégal, principalement en raison du manque de services soins de santé de qualité et un personnel médical bien formé, capable d’effectuer un diagnostic précis des maladies dont souffrent les patients. 
Au cours des dernières années, le gouvernement, avec l'aide de la communauté internationale et des  organisations non gouvernementale a essayé d’éradiquer le paludisme en mettant en œuvre divers mécanismes proactifs et des solutions réactives pour combler le fossé en termes de services de soins de santé et de ressources humaines. Cependant, le taux de mortalité reste  encore très élevé, par exemple dans les zones mal desservies, zones sans soins de santé requis, zone ou la population n’est pas alphabétisée mais aussi dans les zones ou la population a un  faible revenu, etc. La plupart de cas décès de paludisme signalés sont  dus à un diagnostic inexact, parfois incomplet du type exact de paludisme. En revanche, la survenue du paludisme ou sa complication est souvent notée lors d'événements populaires (par exemple des événements religieux tels que comme le Grand Magal de Touba [19]) qui rassemble des milliers de personnes de partout dans le pays pendant une courte période. Au cours de ces événements populaires, les points de santé temporaires sont mis en place pour assister et soigner les personnes malades. Le personnel soignant est parfois uniquement composé de volontaires sans compétences médicales avancées.
Chacun de ces points médicaux peut recevoir et traiter des centaines de patients chaque jour dont  certains d'entre eux sont potentiellement atteints de paludisme. 
Ainsi la nécessité de trouver des outils automatisées pour aider les acteurs médicaux dans le processus de prise de décision après avoir réalisé le diagnostic.
\subsubsection{Approche Proposée.}
Dans cet article, nous présentons les premiers paliers vers une manière efficace de diagnostic automatique du paludisme ou non en fonction des signes et symptômes du patient et les résultats du Test de Diagnostic Rapide(TDR). Nous considérons ce problème de diagnostic comme un problème classique de classification binaire en considérant deux classes: \textquote{Paludisme} et
\textquote{Non-Paludisme}. Ainsi connaissant les données d'un patient, notre objectif principal est de trouver correctement à quelle classe appartient le patient. Pour résoudre ce problème de classification, nous utilisons l’apprentissage automatique avec la  régression logistique comme algorithme de base de notre modèle  de prévision. L'apprentissage automatique a été largement utilisé dans plusieurs domaines (par exemple, l'informatique médicale \cite{Du13}) à des fins diverses, tandis que la régression logistique a démontré son efficacité pour traiter des problèmes de classification binaire tel que le nôtre. Pour le cas de notre étude nous nous intéressons à la prévision du paludisme au Senegal. Pour cela, nous utilisons un grand volume de jeux de données d’enregistrements de patients recueillies pendant le grand Magal de Touba ; l’un des plus grands événements religieux populaire au Sénégal.  Les des différents points de santé installés à cet occasion, à savoir plus de vingt points  reçoivent des centaines de patients. Pour les premiers tests de notre algorithme, nous introduisons une méthode  de préparation de données afin 
\begin{inparaenum}[(i)]
\item d’explorer l'ensemble de données pour une bonne labélisation; 
\item  pour  conserver uniquement les attributs liés au paludisme;
\item  Nettoyer  et transformer les attributs, pour obtenir un jeu de données  numérique composé uniquement que des attributs du paludisme; et 
\item imputer les valeurs manquantes (il y avait beaucoup de valeurs manquantes dans l’ ensemble de données collectées comme le montre la section \ref{data_prep}). 
\end{inparaenum}
La préparation de données a été réalisé en utilisant \emph{OpenRefine} (Google Refine) pour effectuer divers tâches de nettoyage, de traitement, de profilages de l’ensemble des données brutes de patients mais l’algorithme  \emph{missForest} qui est outil robuste pour imputer les données manquantes de différents types; voir la section \ref{data_prep}  pour plus de détails. Les résultats des premiers tests sur un jeu de données  réel et semi-synthétique de patients se sont très prometteurs concernant l’efficacité de l’approche proposée.
\subsubsection{Plan de l’article.}
Le reste de l’article est organisé comme suit. Nous résumons les travaux correspondants à l’imputation des données et les méthodes de classification binaire dans la section \ref{related_work}. Dans la section \ref{data_prep} , nous introduisons un pipeline de préparation de données brutes enregistrées sur les patients recueillies pour la phase de prédiction. Nous présentons ensuite notre modèle de prédiction pour les cas de paludisme dans la section \ref{prediction_model}. Des expériences et analyse de performance de données réelles collectées, ainsi qu’un ensemble de données semi-synthétiques, sont détaillées dans la section \ref{experimentation} avant de conclure dans la section \ref{conclusion}.
 
