% Ceci est samplepaper.tex, un exemple de chapitre présentant le
% Paquet macro LLNCS pour les procédures Springer Computer Science;
% Version 2.20 du 2017/10/04
%
\ documentclass [runheads] {llncs}
%
\ usepackage {graphicx}
\ usepackage {hyperref}
\ renewcommand \ UrlFont { \ color {blue} \ rmfamily }
\ setlength { \ parskip } {0pt}
\ raggedbottom

\ usepackage {amsmath}
\ usepackage {csquotes}
\ usepackage {paralist}
\ usepackage {booktabs}
\ usepackage {mathptmx}
\ usepackage {pgfplots}
\ pgfplotsset {compat = 1.8}
\ usepackage {subfigure}

\ begin {document}
%
\ title {Vers un modèle de prévision efficace \\ des cas de paludisme au Sénégal}
%
% \ titlerunning {titre abrégé de l'article}
% Si le titre du document est trop long pour le titre courant, vous pouvez définir
% un titre abrégé ici
%
\ author {Ousseynou Mbaye \ et Mouhamadou Lamine Ba \\ Gaoussou Camara \ et Alassane Sy}
%
\ authorrunning {Ousseynou Mbaye \ et M. Lamine Ba \ et Gaoussou Camara \ et Alassane Sy}
% Les prénoms sont abrégés dans la tête de liste.
% S'il y a plus de deux auteurs, 'et al.' est utilisé.
%
\ institute {Universit \ ' e Alioune Diop de Bambey, Bambey, Sénégal \\
\ email {firstmidlle.last@uadb.edu.sn}
% \ email {ousseynou.mbaye@uadb.edu.sn} \\
% \ email {mouhamadoulamine.ba@uadb.edu.sn} \\
% \ email {gaoussou.camara@uadb.edu.sn} \\
% \ email {alassane.sy@uadb.edu.sn}
}
%
\ maketitle               % compose l'en-tête de la contribution
%
\ begin {abstract}
Parmi les maladies les plus meurtrières au monde, le paludisme reste un véritable fléau en Afrique subsaharienne 
en particulier. Dans des pays comme le Sénégal, une telle situation est grave en raison du manque de qualité
des services de santé et un personnel bien formé, capable d’effectuer un diagnostic précis des maladies dont souffrent les patients. 
Cela appelle la nécessité de trouver des outils automatisés pour aider les acteurs médicaux dans leur processus de prise de décision.
Dans cet article, nous présentons les premiers pas vers un moyen efficace de diagnostiquer automatiquement l’apparition ou non du paludisme. 
sur la base des signes et des symptômes du patient et des résultats du test de diagnostic rapide. Notre approche de prédiction est construite
sur la fonction de régression logistique. Premiers utilisateurs d'un jeu de données patient réel, ainsi qu'un jeu de données semi-synthétique,
montrer des résultats de performance prometteurs quant à l'efficacité de l'approche proposée.
 
\ keywords {Paludisme   \ et Diagnostic \ et imputation de données \ et modèle de prévision.}
\ end {abstract}
%
%
% Introduction
\ input {intro}

% Travaux connexes
\ input {related_work}

% Préparation des données
\ input {préparation_données}

Modèle de prévision %
\ input {prediction_model}

% D' expérimentation et de résultats
\ input {expérimentation}

% Conclusion
\ input {conclusion}

%
% ---- Bibliographie ----
%
Les utilisateurs de % BibTeX doivent spécifier le style de bibliographie 'splncs04'.
Les références en % seront alors triées et formatées dans le style approprié.
%
\ bibliographystyle {splncs04}
\ bibliography {biblio}
%
\ end {document}
