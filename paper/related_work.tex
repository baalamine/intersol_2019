\section{Related work}\label{related_work}
In this section, we summarize the sate-of-the-art research on Malaria in general, and in particular the
use of machine learning techniques to tackle the various aspects related to one of the 
major healthcare problems worldwide which is Malaria.

% Studies on Malaria
As it is well-known, Malaria is caused by the bite of the female Anopheles, the most dangerous of which
is Plasmodium falciparum. Many early works have been consequently focused on the study of the evolution and
the distribution of the responsible mosquito, mainly with the goal to detect or diagnosis the severity of the 
disease given an infected patient \cite{Fe03,Al09}. Recent research on Malaria have largely adopted machine learning
and showed its ability to solve various aspects of the disease. Most of these machine learning based techniques are 
based on the analysis of blood data obtained from high-definition microscopic screenshots as in \cite{Ku18}. The authors
in \cite{Ku18} propose an unsupervised learning algorithm that detects and determines the types of infected blood cells.
Used prediction approach consists of quantifying the amount of plasmodium parasites in a blood smear. In the same research intuition
of harnessing blood, the Jordan-Elman neural network classifier introduced in \cite{Ha15}, on the other hand, to quickly determine the occurrence 
of Malaria and its severity level as well: the neural network analyzes the features of the blood data of the patients.  
Still using ML, DIAZ et al. have proposed in \cite{Dia09} a semi-supervised algorithm enable to quantify and classify the 
erythrocytes infected by Malaria parasistes through microscopic images. The orginiality of this work comes from its usability
even in the presence of thin blood drandruff infected by falciparum Plasmodium for the quantification and the classificationi tasks.
Besides blood data, sign and symptom records were also used to study Malaria with ML methods. Indeed, decision trees based approach
has been proposed in Nigeria \cite{Ug10} to predict the occurrence of Malaria given diagnostic data. However a decision tree suffers 
from various limitations as a classifier. Indeed it can easily overfit or can be extremely sensitive to small pertubations in data for instance.
Even though we both rely on signs and symptoms, the prediction model in \cite{Ug10} differs from ours on numerous facets: our model is built upon
logistic regression and is trained using also inputs from the quick diagnosis test. In addition, we apply our method in the context of patients living in Senegal. 
An example of previous work that has used logistic regression is that of Farida et al. in \cite{}. The logistic regression is exploited
there for the selection of features in order to construct stable decision trees. The decision trees is then used to predict the severity
criteria of Malaria in the context of Afghanistan. 


In \cite{Pr17}, Pranav et al. propose Malaria likelihood prediction model built on a deep reinforcement learning (RL) agent. 
Such a RL predicts the probability of a patient testing positive for Malaria using answers from question about their household. In 
their model the authors have also dealt with the problem of determining the right questione to ask next as well as the length of the survey, dynamically.
Statistically enhanced rule-based classification model to diagnose malaria has been proposed in \cite{Bb16}. A corresponding prototype 
which incorporates the rules and statistical models have been implemented. The aim of the study was to develop a statistical 
prototype to perform clinical diagnosis of malaria given its adverse effects on the overall healthcare, yet its treatment remains 
very expensive for the majority of the patients to afford.

To the best of our knowledge this is the first work in Senegal that attempts to provide a prediction model for identifying 
the occurrence of Malaria given patient data.  
