\section{Related work}\label{related_work}
In this section, we summarize the sate-of-the-art research on Malaria in general, and in particular the
use of machine learning technique in health Informatics to deal with the different aspects related to the 
major healthcare problems worldwide such as Malaria.

% Studies on Malaria
\paragraph*{Studies on Malaria.}
As it is well-known, Malaria is caused by the bite of the \emph{female Anopheles}, the most dangerous of which
is \emph{Plasmodium falciparum}. Many early works have been consequently focused on the study of the evolution and
the distribution of the responsible mosquito, mainly with the goal to detect or diagnosis the severity of the 
disease given an infected patient \cite{Fe03,Al09}. Recent research on Malaria have largely adopted machine learning
and showed its ability to solve various aspects of the disease. Most oth these machine learning based techniques are 
based on the analysis of blood data obtained from high-definition microscopic screenshots as in \cite{Ku18}. The authors
in \cite{Ku18} propose an unsupervised learning algorithm that detects and determines the types of infected blood cells.
Used prediction approach consists of quantifying the amount of plasmodium parasites in a blood smear. In the same research intuition
of harnessing blood, the Jordan-Elman neural network classifier introduced in \cite{Ha15}, on the other hand, to quickly determine the occurrence 
of Malaria and its severity level as well: the neural network analyzes the features of the blood data of the patients.  
Still using ML, DIAZ et al. have proposed in \cite{Dia09} a semi-supervised algorithm enable to quantify and classify the 
erythrocytes infected by Malaria parasistes through microscopic images. The orginiality of this work comes from its usability
even at the presence of thin blood drandruff infected by falciparum Plasmodium for the quantification and the classificationi tasks.
Besides blood data, sign and symptom records were also used to study Malaria with ML methods. Indeed, decision trees based approach
has been proposed in Nigeria \cite{Ug10} to predict the occurrence of Malaria given diagnostic data. However a decision tree suffers 
from various limitations as a classifier. Indeed it can easily overfit or can be extremely sensitive to small pertubations in data for instance.
Even though we both rely on signs and symptoms, the prediction model in \cite{Ug10} differs from ours in numerous facets: our model is built upon
logistic regression and is trained using also inputs from the quick diagnosis test. In addition, we apply our method in the context of patients living in Senegal. 

% Other ML works in HI
\paragraph*{Other ML works in HI.}
