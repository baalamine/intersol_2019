\section{Related work}\label{related_work}
In this section, we summarize the sate-of-the-art research on Malaria in general, and in particular the
use of machine learning technique in health Informatics to deal with the different aspects related to the 
major healthcare problems worldwide such as Malaria.

% Studies on Malaria
\paragraph*{Studies on Malaria.}
As it is well-known, Malaria is caused by the bite of the \emph{female Anopheles}, the most dangerous of which
is \emph{Plasmodium falciparum}. Many early works have been consequently focused on the study of the evolution and
the distribution of the responsible mosquito, mainly with the goal to detect or diagnosis the severity of the 
disease given an infected patient \cite{Fe03,Al09}. Recent research on Malaria have largely adopted machine learning
and showed its ability to solve various aspects of the disease. Most oth these machine learning based techniques are 
based on the analysis of blood data obtained from high-definition microscopic screenshots as in \cite{Ku18}. The authors
in \cite{Ku18} propose an unsupervised learning algorithm that detects and determines the types of infected blood cells.
Used prediction approach consists of quantifying the amount of plasmodium parasites in a blood smear. In the same research intuition
of harnessing blood, the Jordan-Elman neural network classifier introduced in \cite{Ha15}, on the other hand, to quickly determine the occurrence 
of Malaria and its severity level as well.  

% Other ML works in HI
\paragraph*{Other ML works in HI.}
