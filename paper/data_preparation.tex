\section{Data Preparation}\label{data_prep}
In this section, we detail the data preparation pipeline followed to obtain a proper Malaria dataset for the prediction phase.
We start by presenting the used data cleaning and normalization techniques.


% data cleaning and normalization
\subsection{Data cleaning and normalization}
In order to set up an efficient prediction model for Malaria cases in Senegal, we have relied on a real-world patient dataset
for validation purposes. The dataset has been extracted by more than twelve health points during one of the biggest 
popular religious event in Senegal. Each of these health points dairly receives hundreds of patients, some of them 
suffering from Malaria. 

In detail, the dataset consists of thousands of patient records having each 16 attributes. Some of these
attributes (also knwon as features) comprise personal data about the patient, but also patient signs and symptoms
reported by the doctor who took in charge the patient. The other attributes describe clinical data such as information about the final diagnosis
of the doctor (the disease that the patient suffers from), the income of the quick diagnosis test, and the status (i.e. admission, 
dead or put under observation) of the patient. For privacy concerns and some restrictions in data use, we have disregarded personal data about
the patient during this work. Due to the fact that patient records have been collected manually in registers, we have noticed many inconsistencies 
such as misspellings, same attribute values with different writings (e.g., \textquote{DIARRHEE INFECTIEUSE} and \textquote{INFECTIEUSE DIARRHE}),
 and multi-valued attributes (e.g. sign and symptom reported values). We use OpenRefine \cite{Ku16,openRefine} to first clean and then normalize values in the patient dataset.

OpenRefine (formerly called Google Refine) is a powerful open source tool that allows researchers or scientists to accomplish the data wrangling activity, i.e. 
working with messy data: cleaning it; transforming it from one format into another; and extending it with Web services and external data.    
We used the following methods provided by OpenRefine to pre-treat our raw dataset.

\begin{itemize}
\item \textbf{Text filter function:} text filter enables to expore attribute values, clean them, and to identify those that may have many variants. 
\item \textbf{Transform functions:} OpenRefine provides two different Transform functions: preset transformations for resolving trivial formatting issues like trimming whitespaces
and advanced transformations based on the OpenRefine Expression Language (GREL) to normalize data in batch or split them. This second class of transformations is very useful, especially when the 
number of piece of data values to normalize is very important (doing the same task manually would be time-consuming and prone to errors). For instance, GREL allows to use a simple
regular expression all variants of a symptom in the Symptom column.  
\item \textbf{Cluster and edit function:} Clustering option in OpenRefine also provides users with methods to merge and normalize variations across the dataset. The power of clustering
is that it is able to automatically detect small data variations which follow a certain pattern.  
\end{itemize}

As for the special case of multi-valued attributes such as symptom and diagnosis columns in our raw dataset, we splitted them into multiple values in distinct columns.
Indeed, in the raw dataset information like the symptoms a given patient suffer from were stored in a single column, separated with the special character '+', e.g. \textquote{DOULEUR ARTICULAIRE}
+ \textquote{DOULEUR PELVIENNE} + \textquote{VOMISSEMENTS}.


After this step of data cleaning and normalization, we have proceeded to the extraction of Malaria features.

% Extraction of Malaria features
\subsection{Extraction of Malaria features}
To properly study Malaria, one needs to have a patient dataset with the main features of the disease. 
Unfortunately, some of these Malaria features were not explicitly specified in our raw dataset.
As a result, we have inferred twelve new attributes that better describe the signs and symptoms 
of Malaria according to experts in the health domain. Those new attributes are : \emph{lack of appetite},
\emph{tiredness}, \emph{fever}, \emph{cephalalgia}, \emph{nausea}, \emph{arthralgia}, 
\emph{digestive disorders}, \emph{dizziness}, \emph{chill}, \emph{myalgia}, \emph{diarrhea}, and \emph{abdominal pain}.  
We have then added the new attributes in our dataset and transformed this latter accordingly by filling the value of each new attribute based on the 
list of reported signs and symptoms for every patient.

A medical diagnostic is the results the analysis of the reported signs and symptoms; in general such a diagnostic is further confirmed by a medical test.
Since our raw dataset does not contain only information about patients from Malaria, yielding to records with various diagnostics. For the purposes of 
our study, we replaced any diagnostic that is not Malaria by the class \textquote{Not-Malaria}. At this step of our data preparaton pipeline, we came 
up with a patient dataset that contains required Malaria features. However, our Malaria was not yet complete and ready because of values missingness.
As a last step, we have completed our dataset by using a robust data imputation approach.

% Data Imputation Approach

\subsection{Missing Data Imputation}
As shown in Table \ref{table-missing}, we observed many missing values in our dataset, affecting the majority of the data attributes. Such missing values
should not be ignored as data completeness and quality are very important when dealing with a prediction problem; this could negatively impact the accuracy
of our prediction and should be treat appropriately. One has to note that machine learning relies on complete dataset.
The sources and types of missing values can be various \cite{Al18}. In our context, missingness is \emph{not
completely random} and can be due to an incomplete knowledge of the patient data, the fact that the medical staff do not specify a attribute value when it is
not observed, or a difficulty for the patients to properly describe some piece of information (e.g. related to the signs or symptoms of their diseases) at the
diagnostic time. Since they might have a certain relationship between attribute values for the same patient, or even a correlation between patient records, we 
decide to solve our problem of missing values by using imputation algorithms instead of choosing arbitrary values or removing records with missing values.  

Data imputation is often used in the machine learning field when dealing with missing information. Many algorithms have been proposed in the literature \cite{Si14,Al18},
depending on the nature of the missingness or the type of data. MissForest \cite{Ste12} has been proved to be efficient at the presence of various types (e.g. numerical data,
string, categorical data, ect.) of data simultaneously as in our case.

MissForest relies on Random Forest, a non-parametric prediction method that is able to to deal with mixed-type data and allows for intreactive and non-linear regression effects.
Such a imputation algorithm aims at handling any type of input dataset by minimizing (when possible) assumption about the structural apsects of the data. Technically missForest
solves the missing data problem using an iterative imputation scheme by training a Random Forest on observed values in a first step, followed by predicting the missing values 
and then proceeding iteratively. 

% table of the number of missing values per attribute
\begin{table}[!ht]
\centering
\begin{tabular}{cc}
\textbf{Attribute name} & \textbf{\#missing\_values}\\
\toprule
Lack of appetite &    21068 \\
digestive disorders & 21062 \\
Loss of weight &   21017 \\
Arthragia &     20940\\
Chill &      20925\\
Nausea &     20874\\
Myalgia &    20870\\
Tiredness&   20713 \\
Diarrhea &     20481\\
Vomit &        20051 \\
Abdominal pain &  19770 \\
Dizziness &       19628 \\
Fever &        18245 \\
Temperature &     17636 \\
Arterial pressure &    16924 \\
Cephalalgia &       15370 \\
Diagnostic  &     2875 \\
Quick diagnostic test &  76 \\
\bottomrule
\end{tabular}
\vspace*{.5cm}
\caption{The number of missing values per attribute}\label{table-missing}
\end{table} 
