\section{Data Preparation}\label{data_prep}
In this section, we detail the data preparation pipeline followed to obtain a proper Malaria dataset for the prediction phase.
We start by presenting the used data cleaning and normalization techniques.


% data cleaning and normalization
\subsection{Data cleaning and normalization}
In order to set up an efficient prediction model for Malaria cases in Senegal, we have relied on a real-world patient dataset
for validation purposes. The dataset has been extracted by more than twelve health points during one of the biggest 
popular religious event in Senegal. Each of these health points dairly receives hundreds of patients, some of them 
suffering from Malaria. 

In detail, the dataset consists of thousands of patient records having each 16 attributes. Some of these
attributes (also knwon as features) comprise personal data about the patient, but also patient signs and symptoms
reported by the doctor who took in charge the patient. The other attributes describe clinical data such as information about the final diagnosis
of the doctor (the disease that the patient suffers from), the income of the quick diagnosis test, and the status (i.e. admission, 
dead or put under observation) of the patient. For privacy concerns and some restrictions in data use, we have disregarded personal data about
the patient during this work. Due to the fact the patient records have been collected manually in registers, we have noticed in the dataset issues
such as misspellings, same attribute values with different writings (e.g., \textquote{DIARRHEE INFECTIEUSE} and \textquote{INFECTIEUSE DIARRHE}),
 and multi-valued attributes. We use OpenRefine \cite{Ku16,openRefine} to first clean and then normalize values in the patient dataset.

OpenRefine (formerly called Google Refine) is a powerful open source tool that allows researchers or scientists to accomplish the data wrangling activity, i.e. 
working with messy data: cleaning it; transforming it from one format into another; and extending it with Web services and external data.    


% Extraction of Malaria features
\subsection{Extraction of Malaria features}

% Data Imputation Approach

\subsection{Data Imputation Approach} 
