\section{Prediction Model}\label{prediction_model}
To learn from labelled patient dataset and be able to properly predict 
the occurence or not of Malaria given a new patient, we harness the logistic 
regression function as our \emph{classifier}. 
In this section, we briefy recall the basic of the logistic regression function
and how it can act as a binary classifier. We start by introducing the binary 
classification problem we have to solve in the study.

% binary classification problem for Malaria
\subsection{Binary classification problem}
Let us assume two given classes of Malaria diagnostic: \emph{Malaria} and \emph{Not-Malaria}.
We also consider \textsc{P} and \textsc{C} as respectively the set of patients and a binary prediction model.
A patient \emph{p} in \textsc{P} is defined by a set of pairs $(a_1, v_1), (a_2, v_2), \ldots, (a_n, v_n)$
where $a_i$ and $v_i$, for each $1\leq i\leq n$, respectively corresponds to a given Malaria feature and its associated value defined
as follows.
\begin{equation}
v_i = \left\{
\begin{array}{rl}
1 &\text{if $a_i$ is observed} \\
0 &\text{otherwise} \\
\end{array}
\right.
\end{equation}

\begin{definition}{(Our prediction problem)}
We define our binary classification problem for the prediction of the occurrence
 or not of Malaria on a given patient dataset as a mapping \textsc{C} of every patient p in \textsc{P} 
to one and only one class in \{Malaria, Not-Malaria\}. Formally, we present such a mapping as \textsc{C}: \textsc{P} $\mapsto$ \{Malaria, Not-Malaria\}.
\end{definition}
% Logistic regession function
\subsection{Logistic regression function}
