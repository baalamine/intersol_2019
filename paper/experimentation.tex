\section{Experimentation and results}\label{experimentation}
In this section, we present the performance of our prediction
 model for Malaria occurrecne through an analysis of the results
 of the experimentations we have conducted on real-world datasets
and a semi-synthetic dataset. We start by presenting our experimentation setting.
 
% Experimentation setting
\subsection{Experimentation setting}
We ran tests on three different datasets using the Python Implementation of the logistic regression function.
To impute missing data, we have used the R package of the algorithm missForest.

% Our datasets
\subsubsection{Our datasets}
We collected and used a real-world patient dataset and three of its variants. The description of the data preparation
pipeline to clean, normalize and impute information in our raw  real-world dataset is given in Section \ref{data_prep}.

The first variant of this real-world dataset is obtained by removing records with missing attributes. 
The last variant is a semi-synthetic dataset which has been set up by sampling over our real-world 
dataset. Indeed when we have performed some explanatory analysis on the real-world dataset we have 
observed the dataset was not balanced.

% Implementation of the logistic model
\subsubsection{Implementation of the logistic model}



% Performance measures
\subsection{Performance measures}
In medecine, \emph{sensitivity} and \emph{specificity} are often used, 
while in information retrieval \emph{precision} and \emph{recall} are preferred.
An important distinction is between metrics that are independent on the prevalence
(how often each category occurs in the population), and metrics that depend on the
prevalence – both types are useful, but they have very different properties.

% Analysis of the results
\subsection{Analysis of the results}

