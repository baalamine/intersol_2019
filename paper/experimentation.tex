\section{Experimentation and results}\label{experimentation}
In this section, we present the performance of our prediction
 model for Malaria occurrecne through an analysis of the results
 of the experimentations we have conducted on real-world datasets
and a semi-synthetic dataset. We start by presenting our experimentation setting.
 
% Experimentation setting
\subsection{Experimentation setting}
We ran tests on three different datasets using the Python Implementation of the logistic regression function.
To impute missing data, we have used the R package of the algorithm missForest.

% Our datasets
\subsubsection{Our datasets.}
We collected and used a real-world patient dataset from the different health points which were set during the Grand Magal of Touba in 2016.
We also generated and used two variants of this real-world dataset. The description of the characteristics of our raw real-world dataset, as well as the  data preparation
pipeline that we have proposed in order to clean, normalize and impute information, are given in Section \ref{data_prep}; we refer to this cleaned and complete real-world
patient dataset by \textsc{DT1}. 

We generated the first variant, denoted by \textsc{DT2}, of the raw real-world patient dataset by 
removing records with missing attributes, instead of using an imputation algorithm that will predicts 
values for missing information. Such a variant will help to study the impact of removing records with 
missing values in the prediction accuracy.
 
The second variant, called \textsc{DT3}, is a semi-synthetic dataset which has been set up by using a 
sampling strategy over our raw real-world dataset. Indeed when we have performed some explanatory analysis 
on the real-world dataset have been revealed that the dataset was not balanced, i.e. it shows strong between
class imbalance; the amount of records about patients suffering from Malaria was largely less than the number
of patients that do not suffer from Malaria as shown in Figure \ref{records_class}. Harnessing sampling approaches may enbale to obtain a balanced 
semi-dataset regarding the two classes to predict. To solve our problem of imbalanced dataset, we used the 
algorithm SMOTE \cite{Wa06}, which is a synthetic minority oversampling technique, through its Python implementation
in the  package \emph{imbalanced-learn} \cite{Gu17}. SMOTE consists of predicting a sample of synthetic dataset based on 
the value of the minority class of the targeted class (here the attribute Diagnostic). It randomly chooses the k-nearest 
neighbours of a given record in order to randomly create new observations.
 We have applied an over-sampling of the minority class into our patient 
dataset for generating  a semi-synthetic dataset \textsc{DT3} containing the same number of records for both classes.  

% distribution of records by class
\begin{figure}[h]
\centering
\includegraphics[width=0.6\textwidth]{images/imbalanced_dataset}
\label{records_class}\caption{The number of records by class}
\end{figure} 

% Implementation of the logistic model
\subsubsection{Prediction model setting.}
In order to set up our logistic regression-based classification model, we rely on the Python implementation of the logistic regression in the \emph{sklearn} library\footnote{https://scikit-learn.org/stable/modules/generated/sklearn.linear\_model.LogisticRegression.html}.
This python package defines the logistic regression with the required input parameters, as well as optimization strategies, to properly perform binary 
classification using the best final model. For the purposes of our tests, we have used the following input parameters of the logistic regression:

% input parameters
\begin{itemize}
\item \textbf{random\_state}:
\item \textbf{class\_weight}:
\item \textbf{dual}:
\item \textbf{intercept\_scaling}:
\item \textbf{max\_iter}:
\item \textbf{multi\_class}:
\item \textbf{n\_jobs}:
\item \textbf{penalty}:
\item \textbf{solver}:
\item \textbf{tol:}
\item \textbf{verbose}:
\item \textbf{warm\_start}:
\end{itemize}

As the logistic regression performs a supervised learning we have used 60\% for the training set and 30\% for the test set.

% Performance measures
\subsection{Performance measures}
In medecine, \emph{sensitivity} and \emph{specificity} are often used, 
while in information retrieval \emph{precision} and \emph{recall} are preferred.
An important distinction is between metrics that are independent on the prevalence
(how often each category occurs in the population), and metrics that depend on the
prevalence – both types are useful, but they have very different properties.

% Analysis of the results
\subsection{Experiments and analysis of the results}


\subsubsection{Experiments with DT\_1}
\subsubsection{Experiments with DT\_2}
\subsubsection{Experiments with DT\_3}
\subsubsection{Comparative analysis of the results}
