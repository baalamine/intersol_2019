\section{Introduction}\label{intro}
Malaria is one amongst the most deadly disease in the world, especially in sub-saharan Africa countries such as Senegal.
Malaria is caused by parasitic single-celled microorganisms belonging to the Plasmodium group; it is an infectious
disease which is transmitted to human being through bites from infected female Anopheles mosquitoes. Someone who suffers
from  Malaria may present symptoms that typically include fever, tiredness, vomiting, and headaches. In its severe form,
the disease can cause yellow skin, seizures, coma or death.

\paragraph*{Studied problem and motivations.}
According to the last report about the propagation of Malaria disease around the world, published in November 2017, 216 millions of cases have been 
reported in 2016. As a result, the number of cases has significantly increased when compared to the 211 millions of reported Malaria patients in 2016.
As for the number of death due to Malaria, it does not decrease between 2016 and 2017 (446.000 vs. 445.000) despite the huge effort made by governements
and non-governmental organization to improve healthcare services and the awareness strategies, especially in critical areas. 
When analyzing the statistics above in details, one can easily notice that the burden of the Africa region of the World 
international Health Organization (WHO in short) is colossal. Indeed, 90\% of Malaria cases and 90\% of deaths due to the disease were located in this area in 2016.
More specifically, 80\% of the burden in terms of morbidity is distributed in fifteen countries, all located in Sub-saharan Africa except India. This demonstrates
that Malaria is a real flail in Sub-saharan Africa states and Senegal is not spared at all. We investigate in this study an efficient approach to predict the occurence 
or not of Malaria when a patient has to be diagnosedi using machine learning. Given the patient signs and symptoms, as well as the result from the quick diagnosis test, our solution should be to 
automatically tell if she suffers from Malaria or not with a high accuracy.

Malaria is an acute problem in Senegal  due mainly to the lack of high quality healthcare services and well-formed
staffs able to perform accurate diagnosis of diseases that patients suffer from. Over the past years, the government with 
the help of international organizations have tried to eradicate Malaria by implementing various proactive and reactive solutions 
to fill the gap in terms of services and human resources. However, the mortality rate is still very high, e.g. in underserved areas
or areas without required healthcare needs. Most of these death cases are reported to be caused by inaccurate diagnosis, sometimes 
incomplete leading to a bad prediction of the exact type of Malaria.
On the other hand, Malaria occurence or complication can often occur  during  big events (for instance religious events)
which group thousands of persons from everywhere in the country. During those big events, non-permanent medical points are set in order
to assist and treat ill persons; the staff in a given health point can consist sometimes of only volunteers with no medical skills. Every medical 
point can have to receive hundreds of patients each day with some of them potentially suffering from Malaria. 
This appeals for the need of finding automated tools to help medical actors in their decision making process, and thereby to improve provided 
healthcare services.   

\paragraph*{Proposed diagnosis approach.}
In this paper we present first steps towards an efficient manner to automatically diagnosis Malaria occurence or not based on patient signs and symptoms,
and the outcome from the quick diagnosis test. We define our diagnosis task as a classical binary classification problem by considering two classes: \textquote{Malaria} and \textquote{Not Malaria}.
Given a patient data, our main goal is to properly find to which class the patient belongs. To solve the classification problem we rely onn machine learning  and use the logistic regression
function as the basis of our prediction approach. Machine learning has been used in several domains (e.g. health area) for classification purposes whereas logistic regression has demonstrated its efficiency when the number of features is high. 
 First expermients on a real world 
patient dataset show promising performance results regarding the effectiveness of the proposed approach.

\paragraph*{Paper organization.}The remaining of the paper is organized as follows. We summarize the related work on data imputation and binary classification methods in Section \ref{related_work}.
In Section \ref{data_prep} we introduce the data profiling and imputation techniques used on patient records. We then present our prediction model for Malaria cases in Section \ref{prediction_model}.
Experimentation on real-world datasets are detailled in Section \ref{experimentation} before we conclude in Section \ref{conclusion}. 
