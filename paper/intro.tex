\section{Introduction}\label{intro}
% context and motivations

Malaria is one amongst the most deadly disease in the world, especially in sub-saharan Africa countries such as Senegal.
Malaria is caused by parasitic single-celled microorganisms belonging to the Plasmodium group; it is an infectious
disease which is transmitted to human being through bites from infected female Anopheles mosquitoes. Someone who suffers
from  Malaria may present symptoms that typically include fever, tiredness, vomiting, and headaches. In its severe form,
the disease can cause yellow skin, seizures, coma or death.

According to the last report about the propagation of Malaria disease around the world, published in November 2017, 216 millions of cases have been 
reported in 2016. As a result, the number of cases has significantly increased when compared to the 211 millions of reported Malaria patients in 2016.
As for the number of death due to Malaria, it does not decrease between 2016 and 2017 (446.000 vs. 445.000) despite the huge effort made by governements
and non-governmental organization to improve healthcare services and the awareness strategies, especially in critical areas. 
 


% studied problem and proposed solution










% Papier organization

\paragraph*{Paper organization.}The remaining of the paper is organized as follows. We summarize the related work on data imputation and binary classification methods in Section \ref{related_work}.
In Section \ref{data_prep} we introduce the data profiling and imputation techniques used on patient records. We then present our prediction model for Malaria cases in Section \ref{prediction_model}.
Experimentation on real-world datasets are detailled in Section \ref{experimentation} before we conclude in Section \ref{conclusion}. 
